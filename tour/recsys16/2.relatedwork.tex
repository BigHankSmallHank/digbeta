\section{Related Work}
\label{relatedwork}
% IJCAI'15
\cite{ijcai15} formulated trajectory recommendation as a Orienteering problem,
by optimizing an objective which was a composite of total POI popularity of the recommended trajectory
and the total user interest of the categories of the recommended POIs, the interest of a user at a specific 
category of POI was modelled as the summation of ratios of the visited duration by the user over the average 
duration of all visitors at all POIs of that category. The actual time consumption in a real trajectory, including
both the time spent traveling from one place to another and the time spent at all POIs of the real trajectory,
was used to constrain the time consumed in the recommended trajectory, in addition, with a start place and a 
destination specified, they recommended a trajectory by solving the Orienteering problem using integer programming.
%
The authors formulated a integer programming to solve the itinerary recommendation problem for traveller, the approach based on an essential concept that if a user is more interested in a certain category (e.g. park, museum, etc) of POIs, his/her visit duration should be longer than average in general.


% WSDM'14
\cite{wsdm14} proposed a framework to recommend customized tours (venue's type, visiting order, budget constraint and user statisfaction). They investigated two variants of user statisfaction, i.e. additive benefits and attraction coverage, proved that problems using both user satisfaction functions are NP-hard (reduce TSP to these problems) firstly and then designed two algorithms to solve the two recommendation problems using dynamic programming paradigm, both the time and space complexity of the algorithms are pseudo polynomial/exponential, but they are fast here as the scale of the input in the experimental dataset is small. They also designed an ($1-\epsilon$) approximation algorithm, proposed extensions of their framework, though without evaluating both of them in experiments.
Pros:
take into account both venue's type and user's visit order
formal proof of problem hardness
designed both exact and approximate algorithms
extend framework to incorporate factors appear in real scenarios
Cons:
algorithms evaluated in experiments are still exponential, though efficient in small input scales
dataset used in experiments is not available (a part of another dataset 2010-2011 which could be requested, newer and similar datasets 2011-2012 are freely available)

\cite{tripbuilder15}

\cite{tripplanner15}

\cite{ht14}

\cite{geophoto13}

\cite{travel13}

\cite{transit15}
