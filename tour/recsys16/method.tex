\documentclass{sig-alternate-05-2015}

\begin{document}

%\setcopyright{}
%\doi{}
%\isbn{}
%\conferenceinfo{}
%\acmPrice{\$15.00}
% Author Metadata

\title{Trajectory Recommendation}

%\numberofauthors{}
%\author{
%\alignauthor
%\alignauthor
%\alignauthor
%}

\maketitle

%\begin{abstract}
%\end{abstract}

% The code below should be generated by the tool at
% http://dl.acm.org/ccs.cfm
% Please copy and paste the code instead of the example below. 

%\printccsdesc
%\keywords{}

\section{Proposed Method}


\subsection{Problem Formulation}
\label{sec:formulation}
For a set of Point/Place of Interest (POI) $\mathcal{P}$ and a set of users $\mathcal{U}$,
a trajectory $\mathcal{T}_u$ of user $u$ is an ordered sequence of POIs,
i.e.,
\begin{displaymath}
    \mathcal{T}_u = ((p_1, t_{p_1}^s, t_{p_{1}}^e), \dots, (p_L, t_{p_L}^s, t_{p_L}^e)), 
    u \in \mathcal{U}, 
    p_j \in \mathcal{P}, 1 \le j \le L
\end{displaymath}
where $t_p^s$ and $t_p^e$ is the start and end time of user $u$ at POI $p$ respectively.

Given a set of trajectories $\{ \mathcal{T}_u^i \}, u \in \mathcal{U}$ with visited POIs in $\mathcal{P}$, 
a start POI $p_s$, a destination $p_e$ and an integer $2 < L \le |\mathcal{P}|$,
trajectory recommendation is to recommend the \textit{most likely} trajectory $(p_s, \dots, p_e)$ to user $u$ such that
the number of visited POIs is $L$.


\subsection{Ranking of POIs}
% ranking of POIs focus on modeling the (personalised) preference of POIs
We first produce a ranking of POIs, i.e., $<_{p_i, p_j} \subset \mathcal{P}^2$,
with respect to constraints $(p_s, p_e, L)$ 
using rankSVM with linear kernel and $L2$ loss\cite{lranksvm}, 
i.e.,
\begin{displaymath}
\min_{\mathbf{w}} \frac{1}{2} \mathbf{w}^T \mathbf{w} +
                  C \sum_{(i, j) \in P} \max \left( 0, 1 - \mathbf{w}^T (\mathbf{x}_i - \mathbf{x}_j) \right)^2
\end{displaymath}
where $\mathbf{w}$ is a vector of parameters, 
$C > 0$ is the regularization parameter and 
$\mathbf{x}_i$ is the feature vector of POI $p_i$.
Features used for ranking POIs are:
\begin{itemize}
\item Popularity of a POI, which is defined as the number of distinct users that visited the POI\cite{ht10}.
\item The total number of visits of a POI,
      \begin{displaymath}
          N(p) = \sum_{u \in \mathcal{U}} \sum_i \sum_{p_j \in \mathcal{T}_u^i} \delta(p_j = p)
      \end{displaymath}
\item Average visit duration of POI\cite{ijcai15},
      \begin{displaymath}
          \bar{V}(p) = \frac{1}{N(p)} \sum_{u \in \mathcal{U}} \sum_i \sum_{p_j \in \mathcal{T}_u^i} (t_{p_j}^s - t_{p_j}^e) \delta(p_j = p),
          p \in \mathcal{P}
      \end{displaymath}
\item Whether the category (e.g., structures, entertainment and shopping etc.) is the same as that of $p_s$ (and $p_e$).
\item Distance from POI $p$ to $p_s$ (and $p_e$) using haversine formula \cite{wiki_haversine},
      \begin{displaymath}
      d = 2 R_1 \arcsin \sqrt{ \sin^2 \left( \frac{\Delta \phi}{2} \right) + 
           \cos \phi_p \cos \phi_{p_s} \sin^2 \left( \frac{\Delta \lambda}{2} \right) }
      \end{displaymath}
            where $R_1 = 6371.0088$ km is the mean earth radius \cite{wiki_earth_radius}, 
            $\Delta \phi = \phi_p - \phi_{p_s}$, $\Delta \lambda = \lambda_p - \lambda_{p_s}$,
            and $\phi_p$, $\lambda_p$ are the latitude and longitude of POI $p$ respectively.
\item The difference in popularity of POI $p$ from $p_s$ (and $p_e$),
      i.e., $\Delta Pop = Pop(p) - Pop(p_s)$.
\item The difference in average visit duration of POI $p$ from that of $p_s$ (and $p_e$),
      i.e., $\Delta \bar{V} = \bar{V}(p) - \bar{V}(p_s)$
\item The number of POIs in the expected trajectory, i.e. $L$
\end{itemize}

To generate the target/label of a POI in training set,
we first group trajectories with the same start POI $p_s'$, end POI $p_e'$ and the number of visited POIs $L'$ together,
each group of trajectories form a query,
and the target/label of POI $p$ in a particular query (i.e., trajectory group) are the number of occurrence
in trajectories of that query, 
without counting the occurrence of $p$ as start or end POI of a trajectory.

The ranking scores of POIs are transformed to probabilities using Platt scaling\cite{platt99} by
\begin{displaymath}
    %P(y=1 |p) = \frac{1}{1 + e^{A f(x) + B}, p \in P
    P(p |(p_s, p_e, L)) = \frac{1}{1 + e^{\alpha R(p) + \beta}}, p \in \mathcal{P}
\end{displaymath}
where $R(p)$ is the ranking score of POI $p$, $\alpha$ and $\beta$ are parameters learned from training set\cite{plattnote07}.


\subsection{Factorising Transition Probabilities between POIs}
To deal with data sparsity,
the transition probability between a pair of POIs $(p_i, p_j)$ was factorised into the product of
transition probabilities between the following individual features of that POI pair.
\begin{enumerate}
\item The category of POI, $P(Cat(p_j) | Cat(p_i))$
      is the transition probability from the category of $p_i$ to the category of $p_j$.
\item The popularity of POI, which was first discritized with uniform intervals in log-scale,
      and $P(Pop(p_j) | Pop(p_i))$ is the transition probability from the interval with $Pop(p_i)$ 
      to the interval with $Pop(p_j)$.
\item The total number of visits of POI, similarly, it was first discritized with uniform intervals in log-scale,
      and $P(N(p_j) | N(p_i))$ is the transition probability from the interval with $N(p_i)$ 
      to the interval with $N(p_j)$.
\item The average visit duration of POI, and again, it was first discritized with uniform intervals in log-scale,
      and $P(\bar{V}(p_j) | \bar{V}(p_i))$ is the transition probability from the interval with $\bar{V}(p_i)$ 
      to the interval with $\bar{V}(p_j)$.
\item The neighborhood relationship between $p_i$ and $p_j$,
      which was represented by the geographical clusters of POIs that $p_i$ and $p_j$ reside in,
      and $P(Neighb(p_i, p_j)) = P(c_{p_j} | c_{p_i})$ is the transition probability from the cluster with 
      $p_i$ (i.e., $c_{p_i}$) to the cluster with $p_j$ (i.e., $c_{p_j}$).
\end{enumerate}

Assuming independence between these features,
the transition probability between $p_i$ and $p_j$ can be factorised as follows,
\begin{align*}
    P(p_j | p_i) = & \frac{1}{Z_i} P(Cat(p_j) | Cat(p_i)) \times \\ 
                   & P(Pop(p_j) | Pop(p_i)) \times \\
                   & P(N(p_j) | N(p_i)) \times \\
                   & P(\bar{V}(p_j) | \bar{V}(p_i)) \times \\
                   & P(Neighb(p_i, p_j)), i \ne j, p_i, p_j \in \mathcal{P}
\end{align*}
where $Z_i$ is a normalising constant.

%POIs are grouped into several clusters according to their geographical coordinates using K-means
%to reflect their neighborhood relationships.

\subsection{Recommend Trajectories}
%\subsection{Probabilistic Model incorporating both ranking and transition}
% ranking of POIs focus on modeling the (population) preference of POIs
% MC focus on modeling the transition patterns between POIs
% we can capture both aspects by combining them
% TODO: explain no self-loop in graph G?

For trajectory $(p_{j_1}, \dots, p_{j_L})$ with respect to constraints $(p_s, p_e, L)$, 
where $p_{j_1} = p_s$ and $p_{j_L} = p_e$, 
the likelihood of the trajectory is
\begin{align*}
    \mathcal{L}((p_{j_1}, \dots, p_{j_L}) | (p_s, p_e, L)) =& \prod_{k=1}^L P(p_{j_k} |(p_s, p_e, L)) \times \\
                                                            & \prod_{k=1}^{L-1} P(p_{j_{k+1}} | p_{j_k})
\end{align*}
by taking the logarithm, we get 
\begin{align*}
    \log \left( \mathcal{L}((p_{j_1}, \dots, p_{j_L}) | (p_s, p_e, L)) \right) =& \sum_{k=1}^L P(p_{j_k} |(p_s, p_e, L)) + \\
                                                                                & \sum_{k=1}^{L-1} P(p_{j_{k+1}} | p_{j_k})
\end{align*}

To recommend the \textit{most likely} trajectory with respect to constraints $(p_s, p_e, L)$,
what we need to do is finding a trajectory of maximum likelihood with respect to constraints $(p_s, p_e, L)$,
i.e.,
\begin{align*}
    \text{argmax}_{(p_{j_1}, \dots, p_{j_L})} & \mathcal{L}((p_{j_1}, \dots, p_{j_L}) | (p_s, p_e, L)), \\
                                              & p_{j_1} = p_s, p_{j_L} = p_e, p_{j_k} \in \mathcal{P}, 1 \le k \le L
\end{align*}
which is equivalent to
\begin{align*}
    & \text{argmax}_{(p_s, p_e, L)} \log \left( \mathcal{L}((p_{j_1}, \dots, p_{j_L}) | (p_s, p_e, L)) \right) \\
    &= \text{argmax}_{(p_s, p_e, L)} \left( \sum_{k=1}^L P(p_{j_k} |(p_s, p_e, L)) + \sum_{k=1}^{L-1} P(p_{j_{k+1}} | p_{j_k}) \right)
\end{align*}

One way to find such a trajectory is transforming the above optimization problem to 
finding a shortest path in a weighted graph.

Concretely,
we create a weighted directed graph $G$ with vertices $V$ that corresponds to the set of POIs $\mathcal{P}$ and 
edges $E$ which corresponds to transitions between POIs in $\mathcal{P}$,
the weight of vertex $v_{p}, p \in \mathcal{P}$ is set to the negative logarithm of the ranking probability of POI $p$, 
and the weight of directed edge $(v_p, v_{p'})$ is the negative logarithm of the transition probability from $p$ to $p'$,
i.e.,
\begin{align*}
    w(v_{p})       & = -\log(P(p |(p_s, p_e, L))) \\
    w(v_p, v_{p'}) & = -\log(P(v_{p'} |v_p)), p \in \mathcal{P}
\end{align*}

Thus, compute the most likely trajectory with respect to constraints $(p_s, p_e, L)$ is
equivalent to find a path (not necessarily a simple path) from vertex $v_{p_s}$ to
$v_{p_e}$ with exactly $L$ vertices and minimize the total path weights,
i.e.,
\begin{align*}
    \text{minimize~} & \sum_{i=1}^{L} w(v_i) + \sum_{i=1}^{L-1} w(v_i, v_{i+1}) \\
    \text{s.t.~~~~~} & v_1 = v_{p_s} \\
                     & v_L = v_{p_e} 
\end{align*}
    
The path can be found using dynamic programming, 
with array $A[l, v]$ stores the minimum total weights of path 
that starts at vertex $v_{p_s}$ and ends at vertex $v$ with 
exactly $l$ vertices in path,
array $B[l, v]$ stores the predecessor of $v$ in that path,
and compute a path with $l+1$ vertices is simply the following recursive relations,
\begin{align*}
    A[l+1, v] &= \min_{v' \in Pa_v} \{ A[l, v'] + w(v', v) + w(v') + w(v) \} \\
    B[l+1, v] &= \text{argmin}_{v' \in Pa_v} \{ A[l, v'] + w(v', v) + w(v') + w(v) \} 
\end{align*}
where $Pa_v$ is the parent of vertex $v$ in $G$,
i.e., 
there is a directed edge from $v'$ to $v$, $v' \in Pa_v$.

The minimum path weight is $A[L, v_{p_e}]$,
and the actual path can be found by tracing back from $B[L, v_{p_e}]$,
the complete algorithm to find this path is shown in figure \ref{fig:path},
the sequence of POIs corresponding to vertices in that path is the 
trajectory we would like to recommend with respect to constraints $(p_s, p_e, L)$.

\begin{figure*}
\centering
\begin{tabular}{rl}
\hline
 1:&\textbf{procedure} FindPath$(V, E, p_s, p_e, L)$ \\
 2:&\hspace{10pt} \textbf{for} $v \in V$ \\
 3:&\hspace{20pt}     \textbf{if} $(v_{p_s}, v) \in E$ \\
 4:&\hspace{30pt}         $A[2, v] = w(v_{p_s}, v) + w(v_{p_s}) + w(v)$ \\
 5:&\hspace{30pt}         $B[2, v] = v_{p_s}$ \\
 6:&\hspace{20pt}     \textbf{else} \\
 7:&\hspace{30pt}         $A[2, v] = +\infty$ \\
 8:&\hspace{10pt} \textbf{end for} \\
 9:&\hspace{10pt} \textbf{for} $l=3$ \textbf{to} $L$ \\
10:&\hspace{20pt}     \textbf{for} $v \in V$ \\
11:&\hspace{30pt}         $A[l, v] = \min_{v' \in Pa_v} \{ A[l-1, v'] + w(v', v) + w(v') + w(v) \}$ \\
12:&\hspace{30pt}         $B[l, v] = \text{argmin}_{v' \in Pa_v} \{ A[l-1, v'] + w(v', v) + w(v') + w(v) \}$ \\
13:&\hspace{20pt}     \textbf{end for} \\
14:&\hspace{10pt} \textbf{end for} \\
% //trace back to find the actual path
15:&\hspace{10pt} $path = [v_{p_e}]$ \\
16:&\hspace{10pt} $v = path[0]$ \\
17:&\hspace{10pt} $l = L$ \\
18:&\hspace{10pt} \textbf{repeat} \\
19:&\hspace{20pt}     $path$.prepend$(B[l, v])$ \\
20:&\hspace{20pt}     $v = path[0]$ \\
21:&\hspace{20pt}     $l = l - 1$ \\
22:&\hspace{10pt} \textbf{util} $l < 2$ \\
23:&\hspace{10pt} \textbf{return} $path$ \\
24:&\textbf{end procedure} \\
\hline
\end{tabular}
\caption{Path Finding}
\label{fig:path}
\end{figure*}


%\section{Acknowledgments}

\bibliographystyle{abbrv}
\bibliography{ref}

\end{document}
